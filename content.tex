\documentclass[]{article}

\newcommand{\verbOne}{skipping}
\newcommand{\verbTwo}{VERB}
\newcommand{\verbThree}{VERB}
\newcommand{\verbFour}{VERB}
\newcommand{\verbFive}{VERB}
\newcommand{\verbSix}{VERB}
	
\newcommand{\nounOne}{NOUN}
\newcommand{\nounTwo}{NOUN}
\newcommand{\nounThree}{NOUN}
\newcommand{\nounFour}{NOUN}
\newcommand{\nounFive}{NOUN}
\newcommand{\nounSix}{NOUN}

\newcommand{\adjOne}{lame}
\newcommand{\adjTwo}{ADJ}
\newcommand{\adjThree}{ADJ}
\newcommand{\adjFour}{ADJ}
\newcommand{\adjFive}{ADJ}
\newcommand{\adjSix}{ADJ}

\begin{document}

\section{Mad Libs}

Good moring everyone who's not \adjOne.
Today is a good day to \verbOne~and get things going!
But, first we must come together and \verbTwo~all of our differences before \verbThree.
How do we get there?
Well, it's pretty \adjTwo~if you ask \nounTwo.
You have to \verbFour~\nounTwo~here in this room.
Also, take your \adjFive~computer keyboard and \verbFive~around \nounThree.
Don't forget to \verbFour~the person next to you really \adjThree!

Now, the \verbOne~process is almost \adjTwo!
Next, take everyone's hand and \verbSix~all over \nounFour!
Lastly, is to say goodbye and \verbThree~\nounFive over, and over, and over.

\end{document}
